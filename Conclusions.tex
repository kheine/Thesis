Supersymmetry is among the favoured extensions of the standard model of particle physics and one of the main targets of searches for new physics at the CMS experiment. Since in natural supersymmetric models sparticle masses are expected to be around $\mathcal{O}(1$\tev), the respective phase space can be well explored with \pp collision data obtained during LHC Run I. Here, especially coloured SUSY particles are expected at a high rate which predominantly manifest in final states containing jets and missing transverse momentum. In order to fully exploit all-jet final states in such searches, a precise knowledge of jet-related quantities, like the jet transverse-momentum resolution, is of crucial importance. \\
In this thesis, a measurement of the jet transverse-momentum resolution in dijet events corresponding to data with an integrated luminosity of 19.7\fbinv recorded at $\sqrt{s}=8$\tev in 2012 by the CMS experiment has been presented. Here, systematic limitations of previous analyses have been overcome. This was achieved by an improved treatment of statistical uncertainties by considering correlations among inclusive distributions which then also allowed a revised treatment of systematic uncertainties. Since the systematic uncertainties have been conservatively overestimated in the past, the total precision of the measurement could be significantly improved. Furthermore, the method has been extended to be able to measure the resolution in the forward part of the detector with higher precision. Since no significant trend of the data-to-simulation ratio of the jet transverse-momentum resolution as a function of \ptave was apparent, the ratio is parametrized as a function of $|\eta|$ only and has been determined for $0.0 \leq |\eta| \leq 5.0$. The ratios obtained for the various $|\eta|$ regions increase from $1.08\pm \, 0.03$ in the central region up to $1.40\pm \, 0.06$ for $2.8 \leq |\eta| \leq 3.2$ and drop again for the outermost region $3.2 \leq |\eta| \leq 5.0$ down to $1.06\pm \, 0.19$. A distinct reason for that particular difference between data and simulation could not yet be identified, but in general noise effects, inhomogenities in the detector, miscalibration or inaccurate modelling of the hadronization process are expected to contribute. The determined data-to-simulation ratios can be utilized to adjust the resolution in simulation to match the one observed in data. A publication of the results is currently in preparation~\cite{CMS-JME-13-004}. \\                             
In the second part of this thesis, a search for supersymmetry in proton-proton collisions at $\sqrt{s}=8$~\tev in data corresponding to an integrated luminosity of 19.5~\fbinv has been presented. This is based on events with high jet multiplicities, large values of hadronic energy, missing transverse momentum and no isolated leptons. The main goal of this analysis was to study scenarios arising from supersymmetric models. These involve especially the production and decay of gluinos, light-flavour squarks and gluino-mediated production of third generation squarks. A crucial requirement is a precise estimate of background contributions arising from SM processes. The main focus of the work presented here is the determination of the QCD multijet background which is the most challenging to model for such searches as it requires a precise knowledge of the particle-level jet spectrum. These QCD background contributions are estimated directly from multijet events in data by modelling momentum mismeasurements based on the jet response. A similar approach has been used already in earlier versions of this analysis where the search has been performed inclusive in the jet multiplicity requiring at least three jets. With the extension of the analysis to multijet events, an adjustment of the method to predict QCD multijet background contributions became necessary. A dedicated correction to the existing method has been introduced in order to predict the jet multiplicity correctly. Moreover, the assignment of systematic uncertainties has been revised in order to consider for instance the challenging conditions due to pileup appropriately. In total, the QCD multijet background could be estimated with a precision of approximately $50\%$ in search regions with non-negligible QCD background contributions. The observed number of events in data are consistent with the expected number of events from standard model processes such that exclusion limits are derived for various simplified supersymmetric models. In the context of these simplified models, the production of squarks below 780\gev and that of gluinos up to 1.1--1.2\tev can be excluded at $95\%$ C.L. for LSP masses not exceeding 100\gev. The respective analysis is published in~\cite{Chatrchyan:2014lfa}. \\
In addition to inclusive searches targeting gluinos and squarks, CMS has also performed searches for direct production of supersymmetric top quark partners with data obtained at $\sqrt{s} = 8$\tev. With this data, direct poduction of top squarks decaying into top and LSP could be excluded up to top squark masses of approximately 750\gev for LSP masses below around 100--200\gev. The next run period of the LHC is going to start in 2015 with a centre-of-mass energy of $\sqrt{s} = 13$\tev. These data will open a to the present date unexplored parameter space also for the direct top squark production. In order to extend the mass reach of respective analyses to top squark masses up to the 1\tev scale, suitable selection criteria for such searches have been investigated in the third part of this thesis. Studies presented there are based on events with several jets, large momentum imbalance and no isolated leptons. One key aspect is the application of dedicated algorithms for the identification of decay products emerging from boosted hadronic top quark decays in order to separate possible SUSY signal events from standard model background. Moreover, various different discriminating kinematic variables have been investigated. Finally, the sensitivity of several selections has been compared by deriving the expected exclusion reach for data corresponding to an integrated luminosity of 19.5\fbinv at $\sqrt{s} = 13$\tev. Selection criteria could be identified extending the mass reach of direct top squark searches up to roughly 1\tev for LSP masses below around 300\gev. It has been shown that the proposed selections are also more sensitive than a selection following closely the hadronic search for direct top squark production at $\sqrt{s} = 8$\tev~\cite{CMS-PAS-SUS-13-015}. In addition, it has been demonstrated that such analysis strategies are in general also suitable to study gluino-mediated production of third generation squarks. Thus, this thesis provides a rich variety of strategies to further investigate the question if supersymmetry is realised in nature. 


