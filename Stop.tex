The second run period of the LHC starting in 2015 at a centre of mass energy of $\sqrt{s} = 13$\tev provides the excellent opportunity to further investigate the question, if supersymmetry is realised in nature at the \tev scale. As discussed in Section~\ref{sec:susy_status}, the current limits exclude stop quarks with masses up to around 750\gev for LSP masses below 100\gev in case of direct stop production. Since in case of natural supersymmetry the stop quark mass is expected to not exceed the 1\tev significantly, it is thus of particular interest to probe the stop mass range $\ge 750$\gev for light LSPs. \\
The targeted process of this analysis is illustrated in a diagram in Fig.~\ref{fig:T2tt} below. \\ 
\begin{figure}[!h]
  \centering
  \begin{tabular}{c}
                \includegraphics[width=0.65\textwidth]{figures/T2tt.pdf} 
  \end{tabular}
  \caption{Schematic diagram of the direct pair production of stop quarks in pp collisions with subsequent decay into a top quark and the LSP. Taken from~\cite{bib:CMS:PhysicsResultsSUS}. }
  \label{fig:T2tt}
\end{figure}
   
Here, the pair production of top squarks is shown with subsequent decay into a top quark and the LSP. Since the top quark decays into a b quark and a W boson, the expected final state signature further depends on the decay modes of the W boson. In this analysis, only final states with fully hadronic top decays are considered. Consequently, this analysis targets a jet final state accompanied by missing transverse energy caused by the LSPs. As discussed in Section~\ref{subsec:susy_collider}, background contributions from the SM to such signatures arise mainly from QCD multijet events, \WJets, \ZJets and \ttbar events. \\
In this Chapter, various analysis strategies for a search for top squarks at $\sqrt{s} = 13$\tev are discussed and compared in order to address mass scenarios with large mass differences between the stop quark mass and the LSP. Furthermore, the performance of studied selections is compared to the existing all-hadronic stop search performed by the CMS experiment at $\sqrt{s} = 8$\tev which is published in~\cite{CMS-PAS-SUS-13-015}. In addition, it is interesting to study, if selections that are found to be suited for direct stop production are also sensitive to the gluino-mediated production of third generation quarks. \\

\section{Data samples}
\label{sec:stop_samples}
\begin{table}[!t]
%\fontsize{9 pt}{1.2 em}
%\selectfont
\centering
\caption{Overview of simulated background and signal samples used in the analysis and corresponding production cross sections.}
\makebox[\linewidth]{
\begin{tabular}{llcc}
\multicolumn{4}{c}{} \\
\toprule
   Process &  & $\sigma$ & Precision \\
   \midrule
   \midrule
   \ttbar & & 0.805 [nb] & NNLO \\
   \midrule
   \WJets & \HT = [0, 50]\gev  & 99.92 [nb] & LO \\
          & \HT = [50, 150]\gev  & 15.98 [nb] & LO \\
          & \HT = [150, 300]\gev  & 1.328 [nb] & LO \\
          & \HT = [300, $\infty$]\gev  & 0.169 [nb] & LO \\
   \midrule
   \ZJets & \HT = [0, 100]\gev & 22.0 [nb] & LO \\
          & \HT = [100, 300]\gev & 0.951 [nb]& LO \\
          & \HT = [300, $\infty$]\gev & 0.0396 [nb]& LO \\
   \midrule
   QCD multijet & \HT = [100, 250]\gev & 22\,930 [nb]& LO \\
                & \HT = [250, 500]\gev & 465 [nb] & LO \\
                & \HT = [500, 1000]\gev & 18.66 [nb] & LO \\
                & \HT = [1000, $\infty$]\gev & 0.536 [nb] & LO \\
   \midrule
   \midrule
   Stop-pair production & $m_{\tilde{t}}$ = 600\gev &  0.17460 [pb] & NLO \\
                        & $m_{\tilde{t}}$ = 700\gev &  0.06705 [pb] & NLO \\ 
                        & $m_{\tilde{t}}$ = 800\gev &  0.02833 [pb] & NLO \\ 
                        & $m_{\tilde{t}}$ = 900\gev &  0.01289 [pb] & NLO \\ 
                        & $m_{\tilde{t}}$ = 1000\gev & 0.00615 [pb] & NLO \\ 
                        & $m_{\tilde{t}}$ = 1100\gev & 0.00307 [pb] & NLO \\ 
   \midrule
   \midrule
   Gluino-pair production & $m_{\tilde{g}}$ = 1300\gev & 0.0211 [pb] & NLO \\
                          & $m_{\tilde{g}}$ = 1500\gev & 0.0064 [pb] & NLO \\ 
                          & $m_{\tilde{g}}$ = 1700\gev & 0.0021 [pb] & NLO \\ 
   \bottomrule
\end{tabular}}
\label{tab:stop_samples}
\end{table}    
The studies presented in this chapter are based on simulated samples at $\sqrt{s} = 13$\tev. However, the pileup scenario considered for the samples is the same as for conditions present at $\sqrt{s} = 8$\tev including out-of-time pileup according to a bunch spacing of 50\,ns. Although these are not the pileup conditions expected for $\sqrt{s} = 13$\tev, the selections studied here typically are based on objects with high transverse momenta such that influences from pileup are supposed to be negligible in good approximation. Nevertheless, possible disturbing effects and performance degradation due to pileup necessarily have to be studied as soon as first collision data at $\sqrt{s} = 13$\tev are available. \\
The processes considered in the analysis for background and signal events are summarized in Tab.~\ref{tab:stop_samples}. \WJets, \ZJets and QCD multijet events are generated with MadGraph5 using the PDF CTEQ6L1 while \ttbar events are generated with POWHEG and the MSTW2008 PDF. For all samples the showering process is performed with PYTHIA6 Tune Z2-star. \\
The process \WJets includes decays with $W \rightarrow l \nu$ with up to two jets. Similarly, also the process \ZJets includes up to two jets. Here, decays $Z \rightarrow \nu \nu$ are modelled. For these processes, cross sections are considered at leading order. The scale factors to NLO cross sections are 1.04 and 1.07 for \WJets and \ZJets, respectively. \\
For background processes \WJets, \ZJets and QCD, samples are generated for different \HT selections at generator level. This procedure ensures that especially for the kinematic regions under study which typically concentrate on high-\pt objects a sufficient number of events is generated. \\
In order to study signal events, two different processes are generated. On the one hand direct stop-pair production is considered in which the $\tilde{t}$ always decays to a stable neutralino $\tilde{\chi}_{0}^{1}$ and a top quark with mass $m_{t} = 172.5$\gev including all decay channels of the top quark. These samples are generated with different neutralino masses of 50, 100, 150, 200, 250, 300 and 350\gev. On the other hand gluino-pair production is generated with a decay $\tilde{g} \rightarrow \ttbar \tilde{\chi}_{0}^{1}$. For the top quark only fully hadronic decays are simulated. The cross section given in Tab.~\ref{tab:stop_samples} for gluino-pair production is corrected for the hadronic branching fraction of the top quark. The neutralino mass is fixed to 50\gev. \\
All samples are normalized to an integrated luminosity of 19.5\fbinv which corresponds to the same amount of data recorded at $\sqrt{s} =8$\tev.

\section{Sensitivity of a Basic Selection using \HT and \met}
\label{sec:stop_baseline}
The targeted signature of the direct pair production of stops in the all-hadronic channel is based on jets and missing transverse momentum similar to the search presented in Chapter~\ref{chap:RA2}. Thus, a very similar baseline selection is employed as a basis for further studies. An advantage of such a choice is that a synchronized baseline selection would allow to utilize the same triggers to collect the data. The applied selections are
\begin{itemize}
 \item{Background contributions arising from \ttbar and \WJets events are reduced by rejecting events containing isolated electrons or muons with $\pt > 10$\gev and $|\eta| < 2.5$. These are required to have a good quality track that can be associated with the primary interaction vertex~\cite{CMS-PAS-EGM-10-004, CMS-PAS-MUO-10-002}. The isolation is given as the scalar sum of transverse momenta of PF particles (except for the lepton itself) within a cone of width $\Delta R = 0.3$ for the electron and $\Delta R = 0.4$ for the muon, respectively. This is required to be less than 20\% of the transverse momentum of the electron and less than 15\% of the transverse momentum of the muon.}
 \item{The number of jets (\NJets) is required to be $\ge 3$. \NJets is defined as the number of jets with $\pt > 50$\gev and $|\eta| < 2.5$. This requirement is imposed in order to select multijet events as expected from the two top quarks.}
 \item{The scalar sum of jet momenta (\HT) is required to be $\ge 500$\gev with 
\begin{equation*}
\HT = \sum_{\mathrm{jets}} \pt 
\end{equation*}
for all jets that have $\pt > 50$\gev and $|\eta| < 2.5$. This condition selects events with a large visible energy in the event indicating a high energy scale of the hard interaction.}   
 \item{The missing transverse energy \met calculated from the PF candidates is required to be $\ge 200$\gev. This selection reduces contributions from standard model processes where missing transverse momentum is expected to be small. Especially, QCD multijet background is suppressed. } 
 \item{In order to suppress events where missing transverse energy is mainly arising from jet mismeasurements, as for QCD multijet events, it is required that \met is not aligned with the leading three jets. Thus, events with
\begin{equation*}
\Delta \phi(\mathrm{jet}_n, \met) > 0.5 \;\; \mathrm{for} \;\; n = 1,2 \;\; \mathrm{and} \;\; \Delta \phi(\mathrm{jet}_3, \met) > 0.3
\end{equation*} 
are selected. The value of 0.5 is chosen according to the jet size parameter. However, this is reduced in case of the third jet in order to retain signal efficiency. }
\end{itemize}
Since only simulated events are used, no dedicated event cleaning filters are applied as it is necessary for data (cf. Sec~\ref{subsec:RA2_cleaning}). The selection described here is denoted \textit{baseline} selection in the following. In Tab.~\ref{tab:stop_baseline_cutflow} the event yields of background and two signal processes obtained from simulation normalized to an integrated luminosity of 19.5\fbinv are shown after various requirements together with the statistical uncertainty. It is visible that the \met and $\Delta \phi$ requirement reject most of the background events. After the baseline selection the background is composed quite equally of all four SM processes. Furthermore, in Fig.~\ref{fig:stop_baseline}, the obtained spectra of \HT, \met and \NJets after applying the baseline selection are shown for the SM backgrounds and two selected signal points. The signal points represent mass values of 600\gev and 1100\gev for the stop mass while the LSP mass is in both cases 50\gev. These two signal points illustrate the difference in the kinematic properties of events for low and high stop masses. Typically, the \HT and \met spectrum get harder for higher stop masses while the shape of the \NJets spectrum stays nearly unchanged. \\
\begin{table}[!t]
\fontsize{9 pt}{1.2 em}
\selectfont
\centering
\caption{Event yields and cut flow from MC simulated samples after various requirements described in the text. All numbers are scaled to 19.5\fbinv. Only statistical uncertainties are shown in the table. The signal points are labelled as (X, Y) where X is the top squark mass and Y is the LSP mass.}
\makebox[\linewidth]{
\begin{tabular}{cccccc}
\multicolumn{6}{c}{} \\
  \toprule
   Process & e/$\mu$ veto & jet counting & \HT & \met & $\Delta \phi$  \\
  \midrule
   \ttbar & 10712200 $\pm$ 9170 & 6431190 $\pm$ 7105 & 1274570 $\pm$ 3163 & 34616.8 $\pm$ 521.3 & 16444.9 $\pm$ 359.3 \\
   \WJets & 713523000 $\pm$ 159183000 & 1522960 $\pm$ 20432.5 & 196599 $\pm$ 1259.5 & 25104.5 $\pm$ 295.7 & 12520 $\pm$ 211.6 \\
   \ZJets & 203002000 $\pm$ 408457 & 364126 $\pm$ 4090.1 & 60043.1 $\pm$ 212.4 & 16245.7 $\pm$ 104.9 & 11858.0 $\pm$ 91.1 \\
   QCD multijet & $4.96 \cdot 10^{11}$ $\pm$ $6.89 \cdot 10^{8}$ & $1.01 \cdot 10^{10}$ $\pm$ $8.58 \cdot 10^7$ & $4.43 \cdot 1^{8}$ $\pm$ $3.54 \cdot 10^{6}$ & 109456 $\pm$ 19090 & 20013 $\pm$ 18085 \\
\midrule
   Signal (600, 50) & 2252.7 $\pm$ 8.8 & 1994.7 $\pm$ 8.3 & 1478.3 $\pm$ 7.1 & 1157.9 $\pm$ 6.3 & 1013.1 $\pm$ 5.9 \\
   Signal (1100, 50) & 42.9 $\pm$ 0.2 & 37.3 $\pm$ 0.2 & 36.0 $\pm$ 0.1 & 33.2 $\pm$ 0.1 & 29.2 $\pm$ 0.1 \\
  \bottomrule
\end{tabular}}
\label{tab:stop_baseline_cutflow}
\end{table}    

\begin{figure}[!t]
  \centering
  % \makebox[\linewidth]{
  \begin{minipage}[c]{1.\textwidth}
    \begin{center}
      \includegraphics[width=0.49\textwidth]{figures/Stop_DeltaPhiSelection_HThad.pdf}  
      \includegraphics[width=0.49\textwidth]{figures/Stop_DeltaPhiSelection_MET.pdf} \\
      \includegraphics[width=0.49\textwidth]{figures/Stop_DeltaPhiSelection_N_jets.pdf}
    \end{center}
  \end{minipage}

  \caption{Comparison of selected \HT (\textit{top left}), \met (\textit{top right}) and \NJets (\textit{bottom}) distributions in simulated events found from applying the baseline selection criteria. The signal points are labelled as (X, Y) where X is the top squark mass and Y is the LSP mass.}
  \label{fig:stop_baseline}
\end{figure}
In order to investigate how this baseline selection can be further improved to gain sensitivity to the model of interest the evolution of the signal and background efficiencies is studied when changing specific selections in the analysis. In general, the signal and background efficiencies $\epsilon_\mathrm{sig/bg}$ are defined according to
\begin{equation}
\epsilon = \frac{\mathrm{no. \; of \; selected \; events}}{\mathrm{no. \; of \; all \; events}}
\end{equation} 

\begin{figure}[!t]
  \centering
\makebox[\linewidth]{
  \begin{tabular}{cc}
                \includegraphics[width=0.49\textwidth]{figures/CutScan_DeltaPhiSelection_total_Stop600_LSP50_T2tt_13TeV_HT_MET_MHT_NJets.pdf} & 
                \includegraphics[width=0.49\textwidth]{figures/CutScan_DeltaPhiSelection_TTbar_powheg_13TeV_Stop600_LSP50_T2tt_13TeV_HT_MET_MHT_NJets.pdf} \\
                \includegraphics[width=0.49\textwidth]{figures/CutScan_DeltaPhiSelection_total_Stop1100_LSP50_T2tt_13TeV_HT_MET_MHT_NJets.pdf} &
                \includegraphics[width=0.49\textwidth]{figures/CutScan_DeltaPhiSelection_TTbar_powheg_13TeV_Stop1100_LSP50_T2tt_13TeV_HT_MET_MHT_NJets.pdf}
  \end{tabular}}
  \caption{Evolution of the signal versus background efficiency for a stop mass of 600\gev (\textit{top}) and 1100\gev (\textit{bottom}) where the background is the sum of all backgrounds (\textit{left}) or the \ttbar background (\textit{right}).}
  \label{fig:stop_baseline_cutscan_ht_met_mht_njets}
\end{figure}

for the number of signal and background events, respectively. In Fig.~\ref{fig:stop_baseline_cutscan_ht_met_mht_njets} the development of the signal versus background efficiencies are shown for the stop mass 600\gev and 1100\gev signal points and \ttbar as well as the total background events. The curves are obtained from increasing the cut value for the denoted variable by keeping the selection for all other variables fixed. The curve of a variable with good separation power runs close to the lower right corner. Here, the performance of \HT, \met and \NJets is compared. In addition to \met, also \MHT, as defined in Sec.~\ref{subsec:RA2_baseline}, is shown in this comparison. Both variables show a very similar performance such that in principle both variables could be employed with similar results. However, the performance of \met seems to be slightly better and thus this variable is used in these studies rather than \MHT. Furthermore, as stated above the same trigger as for the analysis presented in Chapter~\ref{chap:RA2} could be used to collect the data. Since this trigger is based on \met, the choice of the offline selections is closer to the online definition when using \met instead of \MHT. As can be seen in Fig.~\ref{fig:stop_baseline_cutscan_ht_met_mht_njets}, \met shows in general the best performance when comparing these variables. Furthermore, also \HT provides a good separation power and is only in case of the stop mass of 600~\gev inferior to \NJets. However, when only \ttbar background is considered, the jet multiplicity is not suitable as discriminating variable since the \NJets spectra of signal and \ttbar background are almost identical. As in general the kinematic features of \ttbar background are closest to the signal, it is of special importance to identify selections that can reduce this background. Consequently, \met and \HT are the preferred variables to distinguish signal and background processes. In the following, an analysis strategy close to the one described in Chapter~\ref{chap:RA2} is pursued. Events selected with the baseline selection are further categorized according to their \HT and \met values in exclusive search regions shown in Tab.~\ref{tab:stop_excl_search_bins}. \\ 
In order to determine the expected exclusion reach, the uncertainties for the individual background processes have to be considered. These are chosen in correspondence to the uncertainties obtained for similar kinematic regimes in the published all-hadronic stop analysis~\cite{CMS-PAS-SUS-13-015}. The respective uncertainties considered for the different processes are
\begin{itemize}
 \item QCD multijet events: 100\%
 \item \ZJets: 50\%
 \item \WJets: 20\%
 \item \ttbar: 20\% + additional 20\% in the high \met search regions
\end{itemize}  
which are meant to be total uncertainties such that the actual statistical uncertainty of the number of simulated MC events is not considered explicitly. Based on the selected event yields and the estimated uncertainties the 95\% confidence-level expected upper limit is calculated as an asymptotic $\mathrm{CL_s}$ limit~\cite{bib:theta}. The obtained exclusion curve is shown in Fig.~\ref{fig:stop_baseline_limit} for the signal strength $\mu$ which is the excluded production cross section divided by the theoretical cross section for direct stop production as function of the stop quark mass. The LSP mass is fixed to 50\gev in order to study the sensitivity of the selection for signal scenarios with large mmass splittings between stop and neutralino mass. A particular mass point can be excluded, if the expected limit drops below one. However, it turns out that this baseline selection cuts and the subsequent binning in exclusive search regions is not yet sensitive enough to probe any of the selected mass points. Thus, possible improvements of the analysis are discussed in the following sections.  

\begin{table}[!t]
%\fontsize{10 pt}{1.2 em}
%\selectfont
\centering
\caption{Exclusive search regions used in the analysis binned in \HT and \met.}
\begin{tabular}{l|c}
\multicolumn{2}{c}{} \\
  \toprule
               & \met [\gev]    \\
  \midrule
   \HT $= 500 - 1000$\gev  & 200 - 400  \\
                          & $>$ 400  \\
  \midrule
   \HT $>$ 1000\gev  & 200 - 400  \\
                 & $>$ 400  \\
                   
  \bottomrule
\end{tabular}
\label{tab:stop_excl_search_bins}
\end{table}    


\begin{figure}[!h]
  \centering
  \begin{tabular}{c}
                \includegraphics[width=0.49\textwidth]{figures/limitplot4BinSel_Baseline.pdf} 
  \end{tabular}
  \caption{Expected 95\% upper limit for signal strength versus $m_{\tilde{t}}$. The LSP mass is chosen to be 50\gev.}
  \label{fig:stop_baseline_limit}
\end{figure}

\section{Sensitivity Improvement using b Tagging}
\label{sec:stop_btagging}
As illustrated in Fig.~\ref{fig:T2tt}, the targeted signal final state involves the presence of bottom quarks emerging from the decay of the top quarks. Thus, a very common thing to possibly enhance the sensitivity of such an analysis is to employ b-tagging techniques to identify the b quarks in the final state. Typical b-tagging algorithms for the identification of b-quark jets used within the CMS experiment have been discussed in Section~\ref{sec:btagging}. In this analysis b-quark jets are identified based on the CSV-algorithm using the medium working point. In Fig.~\ref{fig:stop_baseline_btag} the b-tag multiplicity, \ie the number of b-tagged jets in the event is illustrated. As expected the maximum of this distribution for signal and \ttbar events is around two, since two b-quark jets are expected from the two decaying top quarks. Values above two or b-tagged jets in backgrounds not containing real b-quarks can be explained by the misidentification rate of the tagging algorithm. \\
The resulting distributions for \HT, \met and \NJets after requiring the presence of at least one b-tagged jet in the event are illustrated in Fig.~\ref{fig:stop_baseline_btag_HT_MET_NJets} in App.~\ref{app:stop}. It turns out that all backgrounds except for \ttbar can be significantly reduced when requiring the presence of at least one b-quark jet. This is expected, since the \ttbar background is the only background containing real b quarks. \\
\begin{figure}[!t]
  \centering
  \begin{tabular}{c}
                \includegraphics[width=0.49\textwidth]{figures/Stop_DeltaPhiSelection_N_jets_btagged.pdf} 
  \end{tabular}
  \caption{B-tag multiplicity after applying the baseline selection in simulated events. The signal points are labelled as (X, Y) where X is the top squark mass and Y is the LSP mass.}
  \label{fig:stop_baseline_btag}
\end{figure}
Consequently, the expected exclusion reach of the analysis is supposed to improve when applying the b-tag requirement in addition to the baseline selection. Thus, the same exclusive search regions as defined in Tab.~\ref{tab:stop_excl_search_bins} are used considering the same total uncertainties as described in Sec.~\ref{sec:stop_baseline} for a performance comparison of this improved selection with respect to the baseline requirements. The expected limits for the baseline selection as well as the baseline selection and the b-tag requirement are shown in Fig~\ref{fig:stop_baselinebtag_limit}. It turns out that the b-tag requirement significantly improves the sensitivity of the analysis towards the specified top squark mass range. Especially lower masses could be tested with such an anlaysis strategy. However, since the focus of this analysis is put on higher stop quark masses, it is discussed in the next section how this mass range can be adressed better.  

\begin{figure}[!h]
  \centering
  \begin{tabular}{c}
                \includegraphics[width=0.49\textwidth]{figures/limitplot4BinSel_BaselineBTag.pdf} 
  \end{tabular}
  \caption{Expected 95\% upper limit for signal strength versus $m_{\tilde{t}}$. The LSP mass is chosen to be 50\gev.}
  \label{fig:stop_baselinebtag_limit}
\end{figure}

\section{Sensitivity Improvement using top Tagging}
\label{sec:stop_btagging}
\begin{figure}[!t]
  \centering
  \begin{tabular}{c}
                \includegraphics[width=0.49\textwidth]{figures/Stop_NoCuts_leading_t_ptgen.pdf} 
  \end{tabular}
  \caption{Transverse momentum spectrum of the leading generated top quarks without applying any selection criteria. The signal points are labelled as (X, Y) where X is the top squark mass and Y is the LSP mass.}
  \label{fig:stop_gen_top_pt}
\end{figure} 
In order to gain a better understanding of the kinematics of the investigated final state, the \pt spectrum of the leading generated top quark is illustrated in Fig.~\ref{fig:stop_gen_top_pt} for \ttbar background and two selected signal points without applying any selection criteria. It turns out that the \pt spectrum of the signal is significantly harder than that of the \ttbar background. For instance for a stop mass of 1000\gev the maximum lies at a \pt range around 400-500\gev. As discussed in Sec.~\ref{sec:boosted_tops}, decay products emerging from the decay of a top quark with large transverse momentum do not show up as separated objects in the detector, but merge into jets with large radius parameter. Following Equation~\ref{eq:rule-of-thumb} the opening angle of decay products from a top quark with transverse momentum between 400-500\gev is supposed to be $R = 0.8$. Thus, such topologies are well suited to utilize the top tagging techniques described in Sec.~\ref{sec:boosted_tops}. \\
The performance of the CMS- and the HEP-top-tagging algorithms are investigated based on the 13\tev simulation samples and reviewed in the following. 
\begin{description}
\item {\textbf{Top Tagging Efficiency Studies:}} In order to evaluate the performance of the top-tagging algorithms, the top tagging efficiencies and misidentification rates are derived. While the top tagging efficiency is determined for \ttbar events, the QCD multijet sample is used to measure the misidentification rate. \\
The top tag efficiency is defined as the number of hadronically decaying generated top quarks matched to a top-tagged CA-jet divided by the number of all generated hadronically decaying generated top quarks. A successful match is identified by requiring the $\Delta R$ of a generated top quark and a top-tagged CA-jet to be less than the jet radius parameter which is $R=0.8$ for the CMS top tagger and $R=1.5$ for the HEP top tagger, respectively. The obtained efficiency as function of the transverse momentum of the generated hadronically decaying top quark is shown in the left part of Fig.~\ref{fig:stop_top_eff} for both the CMS and the HEP top tagger. It is visible that the turn-on of the HEP top tagger starts already around 200\gev while the CMS tagger begins to become efficient not before around 400\gev. However, the efficiency of the HEP top tagger lies around 20\% in the plateau region while the plateau efficiency of the CMS top tagger is in general higher with a value of around 25\%. This behaviour results mainly from the different jet sizes and selection criteria which are in case of the HEP top tagger optimized to be sensitive already in the \pt range around 200-300\gev.\\
\begin{figure}[!t]
  \centering
\makebox[\linewidth]{
  \begin{tabular}{cc}
                \includegraphics[width=0.49\textwidth]{figures/Comparison_TopTagEfficiencyGenTopPt_13TeV.pdf} & 
                \includegraphics[width=0.49\textwidth]{figures/Comparison_MistagRate_13TeV.pdf} \\
  \end{tabular}}
  \caption{Top tag efficiency for \ttbar events as function of the transverse momentum of the generated top quark (\textit{left}) and misidentification rate for QCD multijet events as function of the CA-jet \pt (\textit{right}). In case of the CMS top tagger CA-jets with a distance parameter of $R=0.8$ are used while the HEP top tagger is based on CA-jets with $R=1.5$.}
  \label{fig:stop_top_eff}
\end{figure} 
The misidentification rate can be evaluated by dividing the number of top-ttaged CA-jets by the number of all CA-jets. Again, in case of the CMS top tagger the jet distance parameter is $R=0.8$ and for the HEP top tagger it is $R=1.5$. The misidentification rate as function of the respective CA-jet transverse momentum is shown in the right part of Fig.~\ref{fig:stop_top_eff}. Here, a similar feature as for the efficiency curve is observed. The HEP top tagger shows a certain misidentification rate already for lower transverse momenta than the CMS top tagger. However, in general the misidentification rate of the HEP top tagger is significantly smaller with a plateau around 1.5\% compared to the CMS top tagger which shows a misidentification rate of up to 4-5\% in the plateau. \\
Consequently, in analyses which rely on top quarks with moderate transverse momenta and suffer mainly from QCD background the HEP top tagger is a good choice since the efficiency has an early turn-on and the misidentification rate is small. However, when the main background is top-like, as it is the case for top squarks, the more important property is to have an efficiency as high as possible to retain signal efficiency while the actual misidentification rate only plays a minor role. 
\end{description}
\begin{figure}[!t]
  \centering
\makebox[\linewidth]{
  \begin{tabular}{cc}
                \includegraphics[width=0.49\textwidth]{figures/Stop_DeltaPhiSelection_N_jets_toptagged.pdf} & 
                \includegraphics[width=0.49\textwidth]{figures/Stop_DeltaPhiSelection_N_jets_HEPtoptagged.pdf} \\
  \end{tabular}}
  \caption{Top tag multiplicity for the CMS top tagger (\textit{left}) and the HEP top tagger (\textit{right}) after application of the baseline selection. Only CA-jets of the correpsonding jet size with a transverse momentum above 150\gev are considered.}
  \label{fig:stop_top_tag_multi}
\end{figure} 
The impact on the analysis when applying top tag requirements can be studied when looking for instance at the top tag multiplicity. In Fig.~\ref{fig:stop_top_tag_multi} the top tag multiplicity is shown for the CMS top tagger (left) and the HEP top tagger (right) after the application of the baseline selection. Here, only CA-jets of the corresponding jet size with a transverse momentum above 150\gev are considered since the taggers are not supposed to be efficienct for smaller momenta. The distributions exhibit that although also a significant amount of signal events does not have a CA-jet identified as top jet by the respective algorithm the background can be significantly reduced when requiring at least one top-tagged jet. In general the number of signal events with one or two top tags is larger when using the HEP top tagger than for the CMS tagger, but also the number of background events from \ttbar shows a larger value for the HEP top tagger case since more \ttbar events with lower top transverse momenta are selected. \\
In order to test the impact of the top tagging requirement on the search sensitivity, the expected limits are derived again based on the search regions defined in Tab.~\ref{tab:stop_excl_search_bins} with the same uncertainties as assumed in Sec.~\ref{sec:stop_baseline}. The results are illustrated in Fig~\ref{fig:stop_baselinetoptag_limit}. \\
\begin{figure}[!h]
  \centering
  \begin{tabular}{c}
                \includegraphics[width=0.49\textwidth]{figures/limitplot4BinSel_BaselineTopTag.pdf} 
  \end{tabular}
  \caption{Expected 95\% upper limit for signal strength versus $m_{\tilde{t}}$. The LSP mass is chosen to be 50\gev.}
  \label{fig:stop_baselinetoptag_limit}
\end{figure}

It turns out that for the HEP top tagger as well as for the CMS top tagger the usage of a top tag requirement in addition to the baseline selections significantly improves the search sensitivity by a factor of two to three. Hence, mass points up to ($m_{\tilde{t}}, m_\mathrm{LSP}$) = (700, 50) or even (800, 50) can be probed. This selection is also better than the result obtained when adding a b-tag requirement as discussed in Sec.~\ref{sec:stop_btagging} which allows to probe stop masses around 600\gev for an LSP mass of 50\gev. In general, the selection based on the CMS top tagger performs better than the selection involving the HEP top tagger, since the signal to background ratio is higher as discussed above.

\section{Performance Comparison of Various Kinematic Selections}
\label{sec:stop_cuts}

\begin{figure}[!h]
  \centering
  \begin{tabular}{c}
                \includegraphics[width=0.49\textwidth]{figures/limitplot4BinSel_BaselineBTagTopTagTransverseMass.pdf} 
  \end{tabular}
  \caption{Expected 95\% upper limit for signal strength versus $m_{\tilde{t}}$. The LSP mass is chosen to be 50\gev.}
  \label{fig:stop_baselinetoptag_limit}
\end{figure}

\section{Performance Comparison to Selection Based on Published Stop Analysis at $\sqrt{s} = 8$\tev}
\label{sec:stop_pub}
In order to get a better understanding of the quality of the studied selections, a comparison to the analysis criteria used in an analysis based on the same final state that was successfully performed at $\sqrt{s} = 8$\tev is carried out. The respective analysis is published in~\cite{CMS-PAS-SUS-13-015}. \\
The comparison is done by performing the same selections as done in the offical analysis based on the simulated samples discussed in Sec.~\ref{sec:stop_samples}. Jets in this analysis are clustered with the anti-$k_T$ algorithm with a distance parameter of 0.5 and corrected for pile-up effects by applying charged-hadron subtraction. The pre-selection criteria in the analysis are:
\begin{itemize}
 \item Events with isolated electrons and muons with \pt$ > 10$\gev, as described in Sec.~\ref{sec:stop_baseline}, are vetoed.
 \item Events have to have at least five jets with \pt$ > 30$\gev and $|\eta| < 2.4$. The two highest \pt jets further have to have \pt$> 70$\gev while the next two highest jets in \pt must fulfill \pt$ > 50$\gev.
 \item There has to be at least one b-tagged jet in the event based on the CSV algorithm with medium working point. 
 \item The minimum azimuthal angle between the three highest jets and the missing transverse momentum has to be $\Delta \phi(\mathrm{jet}_n, \met) > 0.5$, $n = 1,2$ and $\Delta \phi(\mathrm{jet}_3, \met) > 0.3$.\footnote{The missing transverse energy in the published analysis is denoted with $\pt^\mathrm{miss}$.}% However, just like \met, it is the magnitude of the negative vector sum of the transverse momenta of all particles reconstructed in the event.}
\end{itemize}
The only difference between these selection criteria and the published analysis is related to the lepton veto. The published analysis requires to have no events with identified and isolated electrons and muons with \pt$> 5$\gev while here only events with electrons and muons with \pt$> 10$\gev are vetoed. However, since the 10\gev lepton veto is the same lepton veto requirement as for the other selections studied in this chapter, it is easier to compare the quality of the different selections applied to the sample after selecting the all-jet final state. \\
In addition to those pre-selection requirements, further selection criteria are imposed in order to reconstruct the hadronically-decaying top quarks. The set of five or more jets in the event is separated into all possible combinations of three jets and a remnant which must contain at least one b-tagged jet. These sets are used to reconstruct the two expected top quarks in the event: one is based on one of the trijet combinations and denoted \textit{fully-reconstructed} top while the other is based on the remnant system and referred to as \textit{partially-reconstructed} top. Details on the reconstruction process of the two top quark systems can be found in~\cite{CMS-PAS-SUS-13-015}. In particular this process involves, that the fully-reconstructed top quark has to satisfy the criteria described in Eq.~\ref{eq:hep_1},~\ref{eq:hep_2} and~\ref{eq:hep_3} with $R_\mathrm{min} = 0.85 \cdot (m_W/m_\mathrm{top})$, $R_\mathrm{max} = 1.25 \cdot (m_W/m_\mathrm{top})$, $m_W = 80.4$\gev and $m_\mathrm{top} = 173.1$\gev. If there is more than one trijet system satisfying these criteria the combination with $m^{3-\mathrm{jet}}$ closest to $m_\mathrm{top}$ is selected. The selected remnant system is denoted Rsys. \\
\begin{figure}[!t]
  \centering
\makebox[\linewidth]{
  \begin{tabular}{cc}
                \includegraphics[width=0.49\textwidth]{figures/ReferenceStop_BTag_MET.pdf} & 
                \includegraphics[width=0.49\textwidth]{figures/ReferenceStop_BTag_ref_MT2.pdf} \\
                \includegraphics[width=0.49\textwidth]{figures/ReferenceStop_BTag_ref_Mt3jet.pdf} & 
                \includegraphics[width=0.49\textwidth]{figures/ReferenceStop_BTag_ref_MtRsys.pdf}
  \end{tabular}}
  \caption{Comparison of \met, $M_{T2}$, $M_{T}^\mathrm{3-jet}$ and $M_{T}^\mathrm{Rsys}$ distributions for signal and background events after applying pre-selection criteria. The signal points are labelled as (X, Y) where X is the top squark mass and Y is the LSP mass.}
  \label{fig:stop_ref_plots}
\end{figure} 
After the successful identification of the two top quark systems according to the above mentioned criteria, further topological requirements are used. These are:
\begin{itemize}
 \item \met $> 200$\gev
 \item The variable $M_{T2}$, as discussed in Sec.~\ref{sec:stop_cuts}, is required to be $\ge 300$\gev. It is calculated from the four momenta of the fully- and the partially-reconstructed top quark as well as \met assuming the invisible particles to be massless. 
 \item $(0.5 \cdot M_{T}^\mathrm{3-jet} + M_{T}^\mathrm{Rsys}) \ge 500$\gev. $ M_{T}^\mathrm{3-jet}$ and $M_{T}^\mathrm{Rsys}$ denote the transverse mass of the fully-reconstructed and the remnant system, respectively, which are calculated according to
\begin{equation*}
( M_{T}^\mathrm{3-jet})^2 = (m^{3-\mathrm{jet}})^2 + 2(E_T^\mathrm{3-jet} \, \met - p_T^\mathrm{3-jet} \, \met \, \mathrm{cos} \Delta \phi)
\end{equation*} 
and 
\begin{equation*}
( M_{T}^\mathrm{Rsys})^2 = (m^\mathrm{Rsys})^2 + 2(E_T^\mathrm{Rsys} \, \met - p_T^\mathrm{Rsys} \, \met \, \mathrm{cos} \Delta \phi) \; .
\end{equation*} 
\end{itemize}
A comparison of the \met, $M_{T2}$, $M_{T}^\mathrm{3-jet}$ and $M_{T}^\mathrm{Rsys}$ distributions for signal and background events after the pre-selection is shown in Fig.~\ref{fig:stop_ref_plots}. These exhibit that those topological variables are aible to reject several background events while keeping good acceptance for signal events. \\
In order to probe different points of the parameter space, events are further categorized into four inclusive search regions:
\begin{itemize}
 \item \met $>$ 200\gev, $N_\mathrm{b-jets} \ge 1$ 
 \item \met $>$ 350\gev, $N_\mathrm{b-jets} \ge 1$ 
 \item \met $>$ 200\gev, $N_\mathrm{b-jets} \ge 2$ 
 \item \met $>$ 350\gev, $N_\mathrm{b-jets} \ge 2$ 
\end{itemize}   
Uncertainties in the expected limit calculation performed for each of these categories to compare the reach to the selections discussed above are considered to be the same as those assumed in Sec.~\ref{sec:stop_baseline}. This again makes it easier to compare the actual performance of various selections. From each of the four search regions the expected limit giving the best sensitivity to a specific mass point is considered in the comparison. This is for the scenarios tested here with stop masses ranging from 600 -- 1100\gev and a LSP mass of 50\gev, the selection requiring \met $>$ 350\gev, $N_\mathrm{b-jets} \ge 2$. The actual comparison of the expected limits is illustrated in Fig~\ref{fig:stop_baselinetoptagref_limit}. \\
\begin{figure}[!h]
  \centering
  \begin{tabular}{c}
                \includegraphics[width=0.49\textwidth]{figures/limitplot4BinSel_BaselineTopTagTransverseMassRef.pdf} 
  \end{tabular}
  \caption{Expected 95\% upper limit for signal strength versus $m_{\tilde{t}}$. The LSP mass is chosen to be 50\gev.}
  \label{fig:stop_baselinetoptagref_limit}
\end{figure}

The distributions exhibit that the selection based on the requirements used in~\cite{CMS-PAS-SUS-13-015} shows a quite good sensitivity for low stop masses while the sensitivity drops rapidly towards higher stop masses. However, the selection proposed in Sec.~\ref{sec:stop_cuts} performs better for all mass scenarios.  

\section{Studies of Alternative Selections}
\label{sec:stop_alternatives}

\section{Sensitivity to Gluino-Mediated Stop Production}
\label{sec:stop_gluinos}

\section{Results and Discussion}
\label{sec:stop_results}


