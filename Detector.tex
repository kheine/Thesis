In order to probe the various aspects of the well-established standard model or search for hints of new physics beyond the SM, particle physics experiments preferentially make use of powerful particle accelerators where particles of a certain type are collided in order to probe the constituents of matter and interactions between them. The analyses presented in this thesis are all performed in the context of the CMS experiment located at the Large Hadron Collider (LHC) at CERN near Geneva. \\ 
The first part of this chapter provides an introduction to the LHC. This is followed by an overview of the detector system of the CMS experiment. Afterwards the hitherto periods of collision data taking at the LHC are discussed together with an introduction to the generation of simulated events which are used in the analysis of real data events.  
\section{The Large Hadron Collider}
\label{sec:lhc}
The Large Hadron Collider~\cite{Bruning:782076, 1748-0221-3-08-S08001} is a ring-accelerator designed to provide particle collisions of hadrons. It is built in the tunnel of the former LEP~\cite{LEPdesign} collider 45 -- 170\,m below the ground and has a circumference of 26.7\,km. The LHC is a particle-particle collider and thus composed of two rings with counter-rotating beams. The operation can be performed in different modes with either proton beams or heavy ions like e.g. lead~\footnote{All studies presented in this thesis are based on proton-proton collisions. Thus the operation with heavy ions is not discussed.}. \\
In each beam, protons are grouped together in bunches and accelerated in two evacuated beam pipes using superconducting radio-frequency cavities. With a nominal bunch spacing of 25\,ns the bunch revolution frequency is 40\,MHz. Each of the 2808 individual bunches per beam contains at design conditions $1.15 \times 10^{11}$ protons. In order to bend the beams around the LHC ring superconducting dipole magnets are used with an operation temperature of 1.9\,K. They provide a magnetic field of up to 8.33\,T while additional quadrupole and sextupole magnets are utilized to squeeze and focus the beams.\\  
Before the protons are injected into the LHC they are already pre-accelerated in various smaller accelerators up to a beam energy of 450\,GeV while passing through the injector chain Linac2 -- Proton Synchrotron Booster (PSB) -- Proton Synchrotron (PS) -- Super Proton Synchrotron (SPS). An overview of the accelerator complex at CERN is given in Fig.~\ref{fig:AccComplex}.
\begin{figure}[!tp]
  \centering
  \begin{tabular}{c}
    \includegraphics[width=0.9\textwidth]{figures/AcceleratorComplex.jpg}
  \end{tabular}
  \caption{Illustration of the CERN accelerator complex. Numbers below the names of individual machines indicate the year of their first operation. For ring accelerators also the circumference is given. Taken from~\cite{Christiane:1260465}.}
  \label{fig:AccComplex}
\end{figure}
\\
\\
The main goal of the LHC is to provide proton-proton collisions to the experiments with center of mass energies up to 14\,TeV in order to explore physics processes at novel energy regimes. The expected number of events $N$ for a certain type of process is given by the product of the specific cross section $\sigma$ of that process and the integral $L = \int \mathcal{L}  \, dt$ of the instantaneous luminosity $\cal L$ over time such that
\begin{equation}
  N = \sigma \cdot L . 
  \label{eq:lumi}
\end{equation}
The luminosity is a machine parameter and can be expressed for beams with Gaussian-shaped profiles as  
\begin{equation}
  \mathcal{L} = f \cdot \frac{n_{1}n_{2}}{4 \pi \sigma_{x} \sigma{y}} \cdot F
  \label{eq:lumi}
\end{equation}
with the revolution frequency $f$, the number of particles $n_1$ and $n_2$ contained in the two colliding bunches and the transverse beam sizes $\sigma_{x}$ ($\sigma_{y}$) in the horizontal (vertical) directions. In order to take the inclination of the two beams into account, the geomatrical correction factor $F$ is introduced. The nominal peak luminosity of the LHC is $10^{34} \, \mathrm{cm}^{-2} \, \mathrm{s}^{-1}$.\\
The LHC beams cross at four locations along the ring. At these interaction points the four main experiments of the LHC are located in order to measure the delivered particle collisions. The two high luminosity experiments ATLAS~\cite{det::ATLAS} and CMS~\cite{Chatrchyan:2008zzk, bib:cmsptdr1} are designed for multiple purposes like precision measurements of SM quantities, search for the standard model Higgs Boson or searches for signals indicating new physics processes. The LHCb detector~\cite{det::LHCb} however is a specialised experiment focusing on the measurement of CP violation in the interactions of hadrons containing b-quarks. The only experiment designed especially for the analysis of heavy ion collisions is the ALICE~\cite{det::ALICE} detector with the main emphasis on the physics of strongly interacting matter at extreme energy densities like for instance quark-gluon plasma.

\section{The CMS Experiment}
\label{sec:cms}
The CMS detector is one of the two experiments at the LHC designed to address a multitude of physics questions. In addition to tests of the SM at the TeV scale, studies of the nature of elektroweak symmetry breaking which might show up in the presence of a Higgs boson and searches for so far unknown particles pointing to e.g. new symmetries in nature are the primary targets of these experiments. These ambitious physics goals can only be achieved by fully exploiting the by now unprecedented collision energy and luminosity. Since the total inelastic proton-proton cross-section at a center of mass energy of 14~TeV is expected to be around 100\,mb \todo{Bild + Ref}, the experiments have to deal with an event rate of approximately $10^9$ events per second. This is resulting in high experimental challenges. The CMS detector with its typical cylindrical design of different sub-detector components around the beam line is designed to perfectly meet these particular conditions. A sketch of the CMS detector and the different sub-detectors is shown in Fig.~\ref{fig:CMS}.
\begin{figure}[!tp]
  \centering
  \begin{tabular}{c}
    \includegraphics[width=0.9\textwidth]{figures/CMSDetector.pdf}
  \end{tabular}
  \caption{A perspective view of the CMS detector~\cite{Chatrchyan:2008zzk}.}
  \label{fig:CMS}
\end{figure}
As a typical high-energy particle experiment the CMS detector makes mainly use of tracking detectors and calorimeters to measure particles' momenta, energy depositions and flight directions in order to identify the objects emerging from the particle collisions. Table ... \todo{Performance Table} gives an overview of the performance goals of the various sub-detectors. The overall dimension of the CMS detector are a length of 21.6\,m and a diameter of 14.6\,m resulting in a total weight of 12500\,t. \\  
The following sections comprise a description of the CMS detector and individual sub-detector components focusing on the detector parts most relevant for the analyses presented in this thesis. A detailed discussion of the detector design can be found in~\cite{Chatrchyan:2008zzk, bib:cmsptdr1}.

\subsection{Coordinate Conventions and Kinematic Variables}
\label{subsec:cms_coordinates}
In order to describe the particle collisions, the CMS experiment makes use of a right-handed coordinate system with its origin at the center of the detector at the nominal interaction point. While the z-axis is defined along the direction of the beam, the x-axis points to the center of the LHC ring and the y-axis vertically upwards. In this xy-plane the azimuthal angle $\phi$ is measured where $\phi = 0$ coincides with the x-axis. The polar angle $\theta$ however is defined with respect to the z-axis. A quantity closely related to the polar angle is the pseudorapidity $\eta$ defined as
\begin{equation}
\eta = \mathrm{-ln} \left[\mathrm{tan} \left(\frac{\theta}{2} \right)\right]
\end{equation}
which is widely used in experimental particle physics as rapidity differences are Lorentz invariant. A pseudorapidity $\eta = 0$ corresponds to the direction perpendicular to the beam while $|\eta| \rightarrow \infty$ points along the beam. Based on the pseudorapidity the Lorentz invariant distance between two objects $\Delta$R can be written as
\begin{equation}
\Delta \mathrm{R} = \sqrt{(\Delta \eta)^2 + (\Delta \phi)^2} \, .
\end{equation}
At the LHC the initial conditions of the primary collisions are not known as the specific energy fraction of the proton which each parton carries can not be identified. Thus conservation of the total momentum can not be utilized directly to describe the momentum balance in the final state. However, it is known that the initial particles have no significant momentum orthogonal to the beam axis which is referred to as transverse momentum 
\begin{equation}
\pt = p \cdot \mathrm{sin}\theta \, .
\end{equation}
Thus, momentum conservation in the transverse plane is used to describe the final state conditions. Any difference between the total sum of all transverse momenta and zero is considered as missing energy \met and often exploited to describe undetected particles.

\subsection{Superconducting Magnet}
\label{subsec:cms_magnet}
The CMS experiment makes use of a large superconducting solenoid magnet which is a crucial component of the whole detector design and provides a magnetic field of up to 4\,T. It allows to precisely determine the momenta and charge of charged particles from the bended tracks that they follow in the magnetic field.\\
With a length of 12.5\,m and a diameter of the free bore of 6.3\,m the total cold mass reaches 220\,t. It is made up of a niobium-titanium coil which is winded in 4-layers. This configuration allows a storage of 2.6\,GJ energy at full current. A 10000\,t heavy-weight iron yoke is responsible for the return of the magnetic flux.

\subsection{Inner Tracking System}
\label{subsec:cms_tracker}
The tracking system of the CMS experiment is the innermost part of the detector and installed directly around the interaction point completely contained in the bore of the magnet system. Its' purpose is to precisely measure the trajectories of charged particles arising from the collisions as well as to reconstruct secondary vertices. Due to the location close to the interaction point the tracking system has to cope with a high particle flux crossing the tracker associated with each bunch crossing. Hence high requirements on response time and granularity are set in order to properly identify the particles' tracks. \\
In order to fulfill these tasks the CMS experiment makes use of a tracker design based on silicon detectors. It consists of mainly two components: the innermost part is made of silicon pixel detectors while these are surrounded by silicon strip modules. In total they add up to an active area of 200\,$\mathrm{m}^2$ with a length of 5.8\,m and a diameter of 2.5\,m covering the detector up to $|\eta| = 2.5$. A schematic overview of the whole tracking system is shown in Fig.~\ref{fig:CMS_tracker}. 
\begin{figure}[!tp]
  \centering
  \begin{tabular}{c}
    \includegraphics[width=0.9\textwidth]{figures/fig_cmstracker.png}
  \end{tabular}
  \caption{Sketch of the CMS tracking system in a $rz$-view. Each tracker module is represented by one line. Taken from~\cite{Chatrchyan:2008zzk}.}
  \label{fig:CMS_tracker}
\end{figure}
\begin{description}
 \item \textbf{Pixel Detector:} The pixel detector consists of three barrel layers which extend from 4.4\,cm to 10.2\,cm and two endcap disks on each side. In total there are 1440 pixel modules installed. The size of one pixel cell is 100 x 150 $\mu m^2$ providing similar track resolution quality in $r-\phi$ and $z$ direction. This configuration provides for almost the whole range up to $|\eta| = 2.5$ three precise tracking hits. This is especially important for the reconstruction of secondary vertices.
 \item \textbf{Silicon Strip Tracker:} The silicon strip detector which extends to a radius of 1.1\,m comprises the pixel tracker. The more than 15000 individual strip detector modules are arranged in an inner and an outer detector part. The inner part of the strip tracker is build by the four Tracker Inner Barrel (TIB) layers which are accompanied by the three Tracker Inner Disks (TID) at the end sides. This inner part provides up to four track measurements in the $r-\phi$ plane. The TIB/TID system lies within the Tracker Outer Barrel (TOB) consisting of another six barrel layers while it is complemented by the Tracker EndCaps (TEC) which add another nine disks at each side of the tracking system. This layout provides at least around nine hits within the silicon strip system. 
\end{description}
The tracking system with the desgin described above provides a very good impact parameter resolution and tracking efficiency~\cite{bib:cmstdr:tracker}. The impact parameter resolution is of the order of $<$ 35\,$\mu$m in the plane perpendicular to the beam (for particles with \pt $>$ 10\,GeV) and reaches 75\,$\mu$m in the longitudinal direction. Also the track reconstruction efficiency is expected to show a very good performance. The reconstruction efficiency of high energetic electrons is above 90\%, that of charged hadrons up to 95\% (for \pt $>$ 10\,GeV) and that for muons even better than 98\% in the whole covered region up to $|\eta| = 2.5$ already for muons with very low transverse momenta around 1\,GeV.  

\subsection{Electromagnetic Calorimeter}
\label{subsec:cms_ecal}

\subsection{Hadronic Calorimeter}
\label{subsec:cms_hcal}

\subsection{Muon System}
\label{subsec:cms_muon}

\subsection{Trigger System}
\label{subsec:cms_trigger}

\section{Data Taking and Event Simulation}
\label{sec:data}
