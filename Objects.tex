Particles produced in the pp collisions traverse through the detector and interact with the detector sub-components in a characteristic manner. Thus, it is possible to reconstruct the corresponding event and identify the types of particles which actually emerged from the collision. \\
The approach for the event reconstruction and identification of specific particles used in CMS is discussed in this Chapter. First, the \textit{Particle-Flow (PF) algorithm} used for a global description of the collision event is introduced. Beyond that, in particular the reconstruction of jets is discussed in Section~\ref{sec:jets_reco}. Furthermore, the identification of decays from b-hadrons and boosted top quarks is reviewed in Sections~\ref{sec:btagging} and~\ref{sec:boosted_tops}, respectively.
\section{Global Event Description with the Particle-Flow Algorithm at CMS}
\label{sec:pf_algo}
The CMS experiment introduced the Particle-Flow algorithm for the reconstruction of collision events. This algorithm is designed to identify all stable particles in an event and can be applied to data events as well as to simulated events in an identical manner. Electrons and photons, charged and neutral hadrons as well as muons are thereby distinguishable and all sub-components of the detector are used by the PF algorithm to reconstruct the particles four-momenta. The CMS detector is very well suited for this task. The silicon tracker enclosed by the uniform magnetic field enables a very efficient track reconstruction yielding only a small track fake rate down to small transverse momenta of $150$\,MeV/c. Furthermore, the strength of the magnetic field together with a high ECAL granularity allows photons to be separated from charged-particle energy deposits. A detailed introduction to the PF algorithm can be found in~\cite{CMS-PAS-PFT-09-001}. \\ 
The event reconstruction starts with the identification of fundamental objects in the sub-detectors which are charged-particle tracks, calorimeter clusters and muon tracks. Tracks emerging from charged particles are formed following an iterative tracking algorithm~\cite{Adam:934067}. Starting from an initial seed trajectory, tracks are extrapolated to further tracker layers by taking into account multiple scattering and energy loss in the material following the equations of motion of a charged particle in a constant magnetic field. Each step proceeds with a removal of unambiguously allocated hits from the previous iteration. With this approach a high tracking efficiency as well as a low fake track rate can be achieved which is a crucial requirement for a successful particle-flow event reconstruction. Furthermore, calorimeter clusters are formed in each sub-detector separately based on adjacent calorimeter cells. Neighbouring cells are combined to form clusters when their energy exceeds a pre-defined threshold. \\
A particle traversing through the detector gives typically rise to several of such elementary components so that a dedicated link algorithm is applied in order to connect these elements and form blocks while removing a potential double-counting of the same object in different detector parts. First, charged-tracks are associated to calorimeter clusters, if the extrapolated trajectory matches the cluster within the cluster boundaries. This is done considering effects like gaps and cracks between detector components, uncertainty on the shower position or multiple-scattering. Photons from Bremsstrahlung are considered by extrapolating also tangents of the tracks to the respective energy clusters. In a similar manner ECAL and HCAL clusters can be connected to each other as well by linking clusters in the more granular calorimeter to clusters in the less granular one. At last global muons can be defined by associating charged-tracks from the tracker with muon tracks reconstructed in the muon system. \\
After the identification of such blocks of elements the PF algorithm proceeds to finally create a list of all particles contained in the event applying dedicated quality criteria interpreting the blocks in terms of particles. The identification of muons and a removal of their tracks from the blocks is followed by an assignment of electrons and associated Bremsstrahlung from tracks and linked ECAL clusters. After these have been removed from the list of blocks as well, remaining good quality tracks are considered to be charged hadrons. Their momenta are determined from combining the track momentum and the respective energy in the calorimeter cluster. If the cluster energy largely exceeds the measured momentum from the track beyond the detector resolution, it constitutes a photon and if the excess is larger than the total ECAL energy also a neutral hadron. Finally, remaining ECAL and HCAL clusters not linked to any track give rise to photons and neutral hadrons. \\
The complete set of particles can then be used to derive further objects and quantities like e.g. jets as discussed in Section~\ref{sec:jets_reco}, missing transverse energy $E_{T}^{\mathrm{miss}}$ which is the magnitude of the vector momentum imbalance perpendicular to the beam direction or decay products of tau leptons. More detailled information on the specific quality criteria required for the identification of certain particles is given in the Chapters~\ref{chap:Resolution},~\ref{chap:RA2} and~\ref{chap:Stop} for each analysis presented in this thesis, individually.

\section{Reconstruction of Jets}
\label{sec:jets_reco}

\subsection{Jet Algorithms}
\label{subsec:jets_algos}

\subsection{Jet Types at CMS}
\label{subsec:jets_types}

\subsection{Jet Energy Calibration}
\label{subsec:jets_calib}

\section{Identification of b-Quark Jets}
\label{sec:btagging}
Jets arising from the hadronization of bottom quarks are usually referred to as \textit{b-jets}. As these are existent in many physics processes as e.g. the decay of top quarks, it is of essential importance to be able to identify b jets and distinguish them from jets initiated by gluons or light-flavour quarks. Typically, the identification of b-jets is denoted \textit{b-tagging} and the distinct properties of b quarks are exploited for the identification of the respective jets. In general, hadronic jets arising from the fragmentation of b quarks possess quite large masses, long lifetimes and daughter particles featuring hard momentum spectra. The CMS experiment with the ability to perform a precise charged-particle tracking and lepton identification exploits the specific b jet properties in dedicated b-tagging algorithms for an efficient b-jet identification~\cite{Chatrchyan:2012jua}:
\begin{description}
 \item \textbf{Track Counting (TC) algorithm:} A powerful discriminator for the decay products of a b hadron from prompt tracks is the \textit{impact parameter} (IP) of a track with respect to the primary vertex. Its significance can be computed by taking the ratio of the IP to its respective uncertainty. Tracks in a jet are sorted by decreasing values of the IP significance by the TC algorithm. Depending on whether the IP significance of the second or the third ranked track is chosen as discriminator value the algorithm is denoted \textit{Track Counting High Efficiency} (TCHE) or \textit{Track Counting High Purity} (TCHP) algorithm. 
 \item \textbf{Jet Probability (JP) algorithm:} The JP algorithm extends the simple TC algorithm by connecting the information about the IP from a couple of tracks in the jet. A likelihood is calculated that all tracks of the jet stem actually from the primary vertex. This appraoch can be varied by giving more weight to tracks with the highest IP significance. The maximum of such tracks is four and matches the average number of reconstructed charged particles from the decay of b hadrons. This version is called \textit{Jet B Probability} (JBP) algorithm.
 \item \textbf{Simple Secondary Vertex (SSV) algorithm:} A further useful discriminating feature for b tagging is the presence of a secondary vertex and related kinematic variables like the flight distance and direction which can be determined from the vector between the primary and secondary vertex. The SSV exploits the significance of the flight distance which is given by the flight distance divided by the associated uncertainty. Two different versions of this algorithm exist targeting on the one hand a \textit{High Efficiency} (SSVHE) and on the other hand a \textit{High Purity} (SSVHP). While the SSVHE is based on vertices with at least two associated tracks, the SSVHP uses vertices with more than three tracks. Typically, the efficiency of the algorithm is limited by the reconstruction efficiency of secondary vertices which amounts to about 65\%.  
 \item \textbf{Combined Secondary Vertex (CSV) algorithm:} The CSV algorithm provides an efficient identification of b jets also in cases when no secondary vertex could be recosntructed. It utilizes an approach combining information from secondary vertices as well as track-based lifetime information and thus is able to exceed the efficiency of SSV algorithms. Often, pseudo-vertices can be formed from tracks even when failing the reconstruction of an actual secondary vertex which allows to derive some secondary vertex related quantities. Variables used in the CSV algorithm are e.g. flight distance significance, vertex mass, number of tracks at the vertex, number of tracks in the jet or the IP sigificances for the tracks in the jet. These variables are used to compute two likelihood ratios which can be used to distinguish either c and b jets or light-parton and b jets. 
\end{description}  
Each algorithm determines a discriminator value per jet indicating how b-jet-like a jet behaves. Based on that, working points are defined corresponding to a specific minimum threshold of the discriminator value. These working points are named \textit{loose}, \textit{medium} and \textit{tight} and correspond to a misidentification probability, \ie the probability to identify a non-b jet as b jet, of $10\%$, $1\%$ and $0.1\%$ for an average jet momentum of $80$\gev, respectively. \\
\begin{figure}[!tp]
  \centering 
  \begin{tabular}{cc}
    \includegraphics[width=0.49\textwidth]{figures/figAlgo_Combined_udsgvsb_Efficienies.png} &
    \includegraphics[width=0.49\textwidth]{figures/figAlgo_Combined_cvsb_Efficienies.png} 
  \end{tabular}
  \caption{Performance curves obtained from simulation for the algorithms described in the text. (a) light-parton- and (b) c-jet misidentification probabilities as a function of the b-jet efficiency. Taken from~\cite{Chatrchyan:2012jua}.}
  \label{fig:btagging}
\end{figure}
In order to determine the quality of a particular b-tagging algorithm, typically the misidentification probability as function of the b-tag efficiency is compared for various taggers. Such a performance comparison is illustrated in Fig~\ref{fig:btagging} for the tagging algorithms described above. The misidentification probability is derived separately for light-flavour and gluon initiated jets as well as c-jets. The curves are derived from simulated multijet events using jets with $\pt > 60$\,\gev. For loose working points the b-tag efficiency is around $\approx 80-85\%$ and the JBP algorithm shows the best performance. In case of medium and tight selections, the b-tag efficiency drops to $\approx 45-55\%$ and the CSV algorithm performs best. \\
B-tagging algorithms used for analyses of data obtained at $\sqrt{s} = 8$\tev in 2012 where the TCHP, JP and CSV algorithm~\cite{CMS-PAS-BTV-13-001}. In order to compare the b-tagging performance in data and simulation two different event samples have been selected. On the one hand, an inclusive multijet sample is selected requiring at least one jet with $\pt = 60 - 500$\gev which is dominated by jets from light-flavour quarks and gluons. On the other hand, a sample dominated by top-pair production is selected requiring an electron, a muon and at least two jets with $\pt>30$\gev such that this sample is enriched in b-jets. These samples allow to compare quantities relevant for b-tagging in data and simulation like e.g. the number of secondary vertices or b-tag discriminator values. In general, quantities relevant for b-tagging show a good agreement between data and simulation with deviations within 20\%. Typically, efficiencies and misidentification probabilities are measured as function of jet transverse momentum and pseudorapidity. A potential disagreement between data and simulation can be expressed in terms of \textit{scale factors} and corrected for in analyses such that b-tagging efficiency and misidentification probability in data and simulation match. 

\section{Identification of Boosted Top Quark Decays}
\label{sec:boosted_tops}

\subsection{The CMS Top Tagger}
\label{sec:boosted_tops_cms_tagger}

\subsection{The HEP Top Tagger}
\label{sec:boosted_tops_hep_top_tagger}

%\subsection{Subjet B-Tagging}
%\label{sec:boosted_tops_subjet_b}

%\subsection{N-subjettiness}
%\label{sec:boosted_tops_n_subjettiness}
