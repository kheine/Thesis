In order to probe the various aspects of the well-established standard model or search for hints of new physics beyond the SM, particle physics experiments preferentially make use of powerful particle accelerators where particles of a certain type are collided in order to probe the constituents of matter and interactions between them. The analyses presented in this thesis are all connected to the CMS experiment located at the Large Hadron Collider (LHC) at CERN near Geneva. \\ 
The first part of this chapter provides an introduction to the LHC. This is followed by an overview of the detector system of the CMS experiment. Afterwards the hitherto periods of collision data taking at the LHC are discussed together with an introduction to the generation of simulated events which are used in the analysis of real data events.  
\section{The Large Hadron Collider}
\label{sec:lhc}
The Large Hadron Collider~\cite{1748-0221-3-08-S08001,Bruning:782076} is a ring-accelerator designed to provide particle collisions of hadrons. It is built in the tunnel of the former LEP experiment \todo{Ref: LEP} 45 -- 170\,m below the ground and has a circumference of 26.7\,km. The LHC is a particle-particle collider and thus composed of two rings with counter-rotating beams. The operation can be performed in different modes with either proton beams or heavy ions like e.g. lead~\footnote{The studies presented in this thesis are all based on proton-proton collisions. Thus the operation with heavy ions is not discussed here.}. 

\section{The CMS Experiment}
\label{sec:cms}

\subsection{Coordinate System and Kinematic Variables}
\label{subsec:cms_coordinates}

\subsection{Inner Tracking System}
\label{subsec:cms_tracker}

\subsection{Electromagnetic Calorimeter}
\label{subsec:cms_ecal}

\subsection{Hadronic Calorimeter}
\label{subsec:cms_hcal}

\subsection{Muon System}
\label{subsec:cms_muon}

\subsection{Trigger System}
\label{subsec:cms_trigger}

\section{Data Taking and Event Simulation}
\label{sec:data}
